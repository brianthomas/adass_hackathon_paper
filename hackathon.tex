\documentclass[11pt,twoside]{article}
\usepackage{asp2014}
\resetcounters
\bibliographystyle{asp2014}

\markboth{Thomas et al.}{Time Domain Hackathon}

\begin{document}

\title{The ADASS Time Domain Astronomy Hackathon}
\author{Brian~Thomas,$^1$ Alice~Allen$^2$, Marc~Pound,$^2$ and Peter~Teuben$^2$}
\affil{$^1$NASA HQ, Washington DC  \email{brian.a.thomas@nasa.gov}}
\affil{$^2$Astronomy Department, University of Maryland, College Park}

\paperauthor{Brian~Thomas}{brian.a.thomas@nasa.gov}{0000-0003-1623-9035}{HQ}{NASA}{Washington}{DC}{20742}{USA}
\paperauthor{Alice~Allen}{aallen@astro.umd.edu}{{0000-0003-3477-2845}{Astronomy Department}{University of Maryland}{College Park}{MD}{20742}{USA}
\paperauthor{Marc~Pound}{mpound@astro.umd.edu}{}{Astronomy Department}{University of Maryland}{College Park}{MD}{20742}{USA}
\paperauthor{Teuben~Peter}{teuben@astro.umd.edu}{0000-0003-1774-3436}{Astronomy Department}{University of Maryland}{College Park}{MD}{20742}{USA}


%\aindex{Thomas,~B.}
%\aindex{Allen,~A.}
%\aindex{Pound,~M.W.P.}
%\aindex{Teuben,~P.~J.}

%\ssindex{hack day}
%\ssindex{astronomy!time domain}



\begin{abstract}
In this paper we describe the ADASS XXVIII hackathon, the first associated with an ADASS conference. We provide our motivation, details of the event, and discuss lessons learned. We believe a hackathon associated with the meeting has strong positive value for ADASS and should be considered for future events.
\end{abstract}

\section{Introduction}

A hackathon is an event which seeks to draw together a large group of folks for an intense and extended period of creative programming. Hackathons may be held for a variety of purposes including teaching (Huppenkofthen etal 2018), to draw together a technical community as a social event (Bangert et al 2018) and$\slash$or to draw attention to  solving particular challenges or themes (as found, for example, on popular sites such as Kaggle \url{kaggle.com}).

Our motivation for holding a hackathon associated with the ADASS XVIII meeting was aligned with outreach to interested individuals; we wanted to highlight topical technical problems which the ADASS community might be concerned with and introduce a new generation of rising computer programmers and scientists to the excitement of solving them. We chose the topic area of ``Time Domain Astronomy''  (TDA) \citep{nebot2019} to focus on for this event as it was also one of the themes for this years’ ADASS meeting, as well as aligned well with the interests of the Department that was hosting the event. We allowed a loose definition of TDA, dealing with any astronomical data where time was a parameter. Thus projects for this hackathon could involve variable stars, exoplanets, bodies in the solar system, etc.

\section{Event Organization}

The ADASS hackathon took place the weekend before the ADASS starting on Saturday morning and ending at noon on Sunday with the total event time being 27 hours. We provided a space in the University of Maryland Physical Sciences Complex (PSC) as well as snacks and coffee. The participants were required to attend the introduction and be present for final presentations at 11am on Sunday. Otherwise, they could stay in the PSC building or leave as they desired. A cash award (provided by the City of College Park) was available for the top 3 teams with \$500, \$350 and \$150 being awarded to the first, second and third place teams respectively. The winning team was also provided time to present their hack during the ADASS meeting.
 
We began by having the participants and judges introduce themselves, their backgrounds and interests.  We then introduced the participants to the field of Time Domain Astronomy, providing some general background and challenges in this area. Presentations were given by Charlotte Ward (UMD graduate student), Gerbs Bauer (UMD Research Professor) and Brian Thomas (NASA). We highlighted some datasets which could be applied to solving aspects of the challenges. This was followed by a freely flowing brainstorming session where people could discuss ideas and questions, and potential hacks could be focused. Ideas were placed on sticky notes on a wall. Participants were then allowed a short period of time to form up teams and then start hacking. After another hour or so, each team would present an outline of their hack, potentially allowing members to join another team if skill sets were better suited elsewhere. In our case nobody decided to join another team.

We allowed for a range of project types. Projects could be new analyses/approaches or novel new ways of understanding existing solutions/problems. The final product could be a proof-of-concept app, a plugin to existing code, a storyboard design, or really anything that embodies creative hacking around the Time Domain Astronomy theme. We did not require that the final project to be polished; a good idea which was well fleshed out could also be submitted. A final presentation of a few slides describing the work including the motivation and approach was the only requirement for consideration for a prize.

We used devpost\footnote{\url{https://time-domain-astronomy-6010.devpost.com/}} which helped to structure the hackathon. For example it served as a centralized location from which information could be disseminated,  it encouraged having rules of conduct, judging, assigning judges etc. and provided a discussion board which we primarily used to distribute ideas and answer participant questions.

Hackathon rules can be summarized as follows:

\begin{quote}
$\bullet$ Each participant belongs to one team and one final submission, but it is allowed to switch teams. Team makeup is not final until the presentation. The maximum team size was 5.

$\bullet$ Only 1 submission per team.

$\bullet$ Code of Conduct. We did not tolerate harassment of hackathon participants in any form, including, but not limited to, harassment based on gender identity and expression, age, sexual orientation, disability, physical appearance, body size, race, ethnicity, nationality, religion, political views, previous hackathon attendance or lack of computing experience or lack of chosen programming language or tech stack. Sexual language and imagery was not appropriate at any point in the hackathon including in software hacks, social media, talks, presentations, or demos.
\end{quote}

Hackathon participants violating any of these rules could be sanctioned or expelled from the hackathon at the discretion of the hackathon organisers.

\section{Participants}

It was set up as a community hackathon and our event attracted students, professional hackathonners and ADASS participants who formed teams (see below). Members of the local and permanent organizing community of ADASS judged the hackathon.  Out of the 34 registrations, 6 were present but not playing (being part of the organization or just cheerleading), and 9 did not show up.

\subsection*{Judges, Organizers and Teams}

The session was organized by Peter Teuben, Brian Thomas, Alice Allen, Marc Pound and Elizabeth Warner. Our judges were Alice Allen, Gerbs Bauer, Andy Harris, Nuria Lorente, Ada Nebot and Brian Thomas. The 7 teams which participated are summarized in Table 1. We have also noted which teams won which prizes.

\begin{table}
\caption{Hackathon Teams} \label{tab:title} 
\begin{center}
\begin{tabular}{|l|l|} \hline 
{\bf Team Members} & {\bf Project Name } \\ \hline\hline 
Sarah Frail and Patrick Shan & Morpheus - Near Earth Objects \\ \hline 
 & Visualization \\ \hline
	Marco Lam &  Drag and drop ensemble (2$^{nd}$ Prize) \\ \hline 
Paul Ross McWhirter and & Auto periodogram selection using MC \\ 
	Josh Veitch-Michaelis &  (3$^{rd}$ Prize) \\ \hline 
Timothy Henderson and Matt Graber &  Solar Activity Viewer \\ \hline 
Thomas Boch, Matthieu Baumann, & Music of Light curves \\
	and Siddha Mavuram & (1$^{st}$ Prize) \\ \hline
Kyle Kaplan, Sankalp Gilda, & ML on ZTF pipeline \\ 
Hayden Hotham, Steve Gambino & \\ 
and Abbie Petulante & \\ \hline 
Kevin Cai, Kael Lenus, James Zhou, & Fixed and Variable Time Kepler \\ 
and Justin Otor &  Viewer in WWT \\ \hline\hline 
\end{tabular}
\end{center}
\end{table}

\section*{Lessions Learned}


This was the first such event of this type for ADASS it was somewhat of an experiment. There were several lessons learned.

\begin{enumerate}

	\item \textit{Provide a list of interesting problems and related clean data.} Doing so helps to bootstrap project ideas as not all participants will have enough domain background to start quickly. Because the event was so short, it was helpful to provide microservices and point to datasets which were more or less cleaned and ‘ready to go’ for projects directed at these problem areas.

	\item \textit{Develop a marketing plan.} We could have done a better job to garner interest in the event. We posted to a community bbs, a UMD subreddit, posted paper flyers in campus science and engineering buildings, and contacted student groups and faculty to help spread the word. However, we did not have a coordinated campaign that included social media and messaging targeted for specific dates and groups (e.g., “Save The Date” emails). Nor was the hackathon mentioned in the ADASS registration form.  A competing, large, all-women hackathon \url{https://gotechnica.org/}  held the same weekend on campus also affected our enrollment.

	\item \textit{Venue (location and time) is important.} The university was a good choice because of easy access to rooms, wifi, and food choices. Holding the hackathon at a large academic institution ensured that it would be easy for younger participants (undergrads) to attend, as did holding the event over a weekend to avoid conflicting with classes.

	\item \textit{Have an assessment tool$\slash$strategy.} An exit survey or ending discussion with participants can help improve subsequent hackathons. We failed to take advantage of the opportunity to engage either the participants or the ADASS audience at the session where winning projects were presented about perceived problems and good aspects of our event.

	\item \textit{Narrow the range of participant experience.} Future organizers should consider either limiting participation to non-professionals, or group the participants and awards into professional and non-professionals.  It is somewhat unfair to have less experienced coders compete against professional specialists and possibly contrary to the avowed desire to use this event to advertise our field of work to outsiders.

	\item \textit{Time management is crucial.} Scheduling a conference event right at the end of the hackathon was problematic, and not tightly managing the final presentation time well and similar issues became important and detracted from the event. This will be particularly important in other events which have larger participation.

\end{enumerate}

\section*{Conclusions}

A community lives and dies by how well it nurtures the next generation. Folks enter the ADASS community by a number of means but typically by being either scientists who become attracted to the technical challenges of writing the software or as computer engineers and programmers who find the science use cases particularly interesting. We are not aware of any organized means to train the next generation of ADASS workers; there are no formal degree programs in “Astronomy Software.” As such, our community has taken a somewhat laissez faire approach to training the next generation and this may lead to a future deficit in skilled professionals willing to work in our field.  More and more our communities skills are being found useful in application elsewhere; for example, many ADASS workers can easily become highly sought after Data Scientists.

Hackathons are a step towards being more proactive in our outreach and provide an ideal means to encourage and interest a younger group of programmers in the complex and interesting challenges which our community tackles. We found a number of lessons in hosting this event, but no showstoppers, and a good deal of positive goodwill was generated. Based on our experience we heartily recommend that future ADASS events include hackathon events.

\acknowledgements We would like to thank the City of College Park\footnote{\url{www.collegeparkmd.gov}} for providing the prize money, Vigilante Coffee\footnote{\url{vigilantecoffee.com}} for supplying much needed coffee, ASCL\footnote{\url{ascl.net}} for providing snacks and the University of Maryland Astronomy Department\footnote{\url{www.astro.umd.edu}} for hosting the hackathon.


\bibliography{H1}

% photos supposed to ago at end
% i'm counting 17 in here, but we list 19 ... o well.
% bt: thats because not everyone showed for the photo at the end
%
\bookpartphoto[width=1.0\textwidth]{adass-hackathon-group.eps}{Most of the hackathon team members. Some of the winners are holding envelopes. (Photo: Brian Thomas)}

\end{document}
