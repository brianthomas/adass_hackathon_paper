\documentclass[11pt,twoside]{article}
\usepackage{asp2014}
\resetcounters
\bibliographystyle{asp2014}

\markboth{Thomas et al.}{Time Domain Hackathon}

\begin{document}

\title{The ADASS Time Domain Astronomy Hackathon}
\author{Brian~Thomas,$^1$ Alice~Allen$^2$, Marc~Pound,$^2$ and Peter~Teuben$^2$}
\affil{$^1$NASA HQ, Washington DC  \email{brian.a.thomas@nasa.gov}}
\affil{$^2$Astronomy Department, University of Maryland, College Park}

\paperauthor{Brian~Thomas}{brian.a.thomas@nasa.gov}{}{HQ}{NASA}{Washington}{DC}{20742}{USA}
\paperauthor{Alice~Allen}{aallen@astro.umd.edu}{}{Astronomy Department}{University of Maryland}{College Park}{MD}{20742}{USA}
\paperauthor{Marc~Pound}{mpound@astro.umd.edu}{}{Astronomy Department}{University of Maryland}{College Park}{MD}{20742}{USA}
\paperauthor{Teuben~Peter}{teuben@astro.umd.edu}{0000-0003-1774-3436}{Astronomy Department}{University of Maryland}{College Park}{MD}{20742}{USA}

%\aindex{Thomas,~B.}
%\aindex{Allen,~A.}
%\aindex{Pound,~M.W.P.}
%\aindex{Teuben,~P.~J.}

%\ssindex{hack day}
%\ssindex{astronomy!time domain}


\begin{abstract}

The ADASS Hackathon was set up as a community hackathon, attracting
students, professional hackathonners and ADASS participants to form
teams to work out user selected problems in Time Domain Astronomy
(TDA), one of the themes of this years' ADASS. The hackathon took
place the weekend before ADASS, and the winning team presented their
hack during the TDA session at ADASS.


\end{abstract}



\section{Introduction}

There are currently many ways to collaborate programming efforts to solve interesting problems.
Sometimes this brings together people with different skillsets. We choose the hackathon



We used devpost\footnote{\url{https://time-domain-astronomy-6010.devpost.com/}}
which structures a hackathon a bit. For example, it encourages rules of conduct,
judging, assigning judges etc.  It creates a discussion board, which we primarely used to
distribute ideas.

We started on Saturday morning, with some introductory talks and the opportunity to form teams.
Since the ADASS conference started Sunday at 13:00 with their tutorials, we had to wrap up
the Hackathon by then, so this was only a ${1 {1\over 2}}$ day hackathon.


\section{Introduction Talks}

Short introduction talks were given by three scientists: Brian Thomas (also master of ceremony),
Charlotte Ward (UMD graduate student) and and Gerbs Bauer (UMD Research Professor)
giving ideas what hacking could be done. 


We will need (links to) datasets. But lots of pretty pictures to excite
them.  After this introduction the moderator will gather ideas
from the audience, they are written on stickies, and put on a big
board.  Then the moderator will group them a bit, by topic (or
whatever method).  Participants then come to the front, talk to each
other, talk about the ideas on the stickies, and self-organize in
groups.

Those groups then meet for about 1/2 - 1 hour to come up with a
plan. These plans are then presented in front of the audience, who can
comment.  Also at this stage we allow people to change groups, but not
in large hoards. E.g. a group could say we really need a UI designer,
and maybe somebody will walk over.

By this time it's coming up close to lunch. The groups will now start
working in all earnest, take lunch and dinner whenever they want, and
continue work. Sleep is optional, but highly encouraged. In the
morning 3 more hours are devoted to get the ideas presentable.
Presentation are at noon, with the awards at 1pm. [caution: ADASS 1st
  tutorials also at 1pm, so we may need to adjust if there is overlap]

It should be stressed that no finished products are needed (but would
be nice) for the presentation. Mockups plus realistic estimates how
they are to be implemented are acceptable too. All materials should
be in a public github (or likewise) repo.

The winner should present at the Time Domain Astronomy session at
ADASS (time TBD), but if presenters are not able to come, movie of
their presentation could be shown.  A VIP will also reward the winner
with prize money .

\section{Ideas}

We had encouraged some ideas on the devpost website, before and during the
introduction talks.


\section{Participants}

We had 34 participants signed up, 25 showed up.
Six did not play, as they were part of the organization/jury or for
other reasons, leaving us with 19 participants, forming 7 teams

Peter Teuben, 
Perry Publico, 
Bryen Xie, 
Acheev Bhagat, 
Hayden Hotham, 
CAKS42, 
Matt Graber, 
Timothy Henderson, 
Marco Lam, 
Sarah Frail, 
Alice Allen, 
Sankalp Gilda, 
Marc Pound, 
Justin O, 
Baptiste Cecconi, 
Siddha Mavuram, 
Josh Veitch-Michaelis, 
Luca Rizzi, 
Thomas Boch, 
Steve Gambino, 
Abbie Petulante, 
Matthew Brown, 
xiuqin, 
Kevin Cai, 
Paul Ross McWhirter, 
Patrick Shan, 
Fanyin, 
Arnav Shivansh, 
Hyekang Joo, 
Kael Lenus, 
James Zhou, 
Brian Thomas, 
Matthieu Baumann, 
Kyle Kaplan



\section{Teams}

In order or presentations, the following teams presented, where we awarded
a hard contensted 1$^{st}$, 2$^{nd}$, and 3$^{rd}$ Prize.

\subsection*{Team 1: Morpheus}
Sarah Frail and Patrick Shan

\subsection*{Team 2: Drag and drop ensemble (2$^{nd}$ Prize)}
Marco Lam

\subsection*{Team 3: auto periodogram apes selection using MC (3$^{rd}$ Prize)}

Paul Ross McWhirter and Josh Veitch-Michaelis

\subsection*{Team 4: Solar Activity Viewer}

Timothy Henderson and Matt Graber

\subsection*{Team 5: Music of Light curves (1$^{st}$ prize)}

Thomas Boch, Matthieu Baumann, and Siddha Mavuram


\subsection*{Team 6: ML on ZTF pipeline}


Kyle Kaplan, Sankalp Gilda, Hayden Hotham, Steve Gambino, and Abbie Petulante


\subsection*{Team 7: Fixed and Variable Time Kepler Viewer in WWT}

Kevin Cai, Kael Lenus, James Zhou, and Justin Otor


\citep{foo}.

\subsection*{Non players}

Peter Teuben,
Alice Allen,
Marc Pound,
xiuqin,
Brian Thomas,
Matthew Brown

\subsection*{No shows}

Perry Publico,
Bryen Xie,
Acheev Bhagat,
CAKS42,
Baptiste Cecconi,
Luca Rizzi,
Fanyin,
Arnav Shivansh,
Hyekang Joo

% ack before bib
\acknowledgements We would like to thank the City of College Park to
provide the prize money, Vigilante Coffee for supplying with much
needed coffee, ASCL for providing snacks and the Astronomy Department
of hosting the hackathon.


\bibliography{H1}

% photos supposed to ago at end
\bookpartphoto[width=1.0\textwidth]{adass-hackathon-group.eps}{Hackathon participants. Some of the winners holding envelopes. (Photo: Brian Thomas)}



\end{document}
